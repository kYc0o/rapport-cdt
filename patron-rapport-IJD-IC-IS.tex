\documentclass[a4paper, twoside]{article}

\usepackage[T1]{fontenc}
\usepackage{fullpage}
\usepackage[french]{babel}
\usepackage[latin1]{inputenc}
\usepackage{color}
\usepackage{hyperref}

\newenvironment{notes}[0]{\color{blue}}{\color{black}}

\title{\vskip 5cm {\huge
    Rapport d'activité 
      %~\footnote{}
      %/ Rapport final
      %~\footnote{Si l'année en cours est la deuxième année de contrat}
      \\
      D\'eveloppement open-source et maintenance d'un OS embarqu\'e pour l'Internet des objets
  }
}
\author{\large \sl \bf
  Francisco Javier Acosta Padilla \\
  Ingénieur spécialiste
}
\date{
  Du 1 janvier 2016 au 26 septembre 2016
}

\begin{document}
\maketitle

\newpage

%%==============================================================================
%%==============================================================================
\section{Contexte}


\subsection{Introduction}


Cette ADT a été lancée dans le cadre du projet de développement d'un système
d'exploitation pour l'internet des objets, RIOT.\\

Le développement de RIOT a démarré en 2013 et compte à ce jour plus de 100 développeurs partout dans le monde.
Ce projet est principalement soutenu par Inria, Freie Universität à Berlin et Hochschule für Angewandte Wissenschaften à Hamburg.

Le code source est disponible sur
Github\footnote{\url{https://github.com/RIOT-OS/RIOT/}}, sous licence LGLPv2,
qui permet à toute personne son utilisation, modification et distribution.

Ce code est essentiellement écrit en C et utilise l'outil
Make pour la compilation. D'autres langages comme Python ou C++ sont également
utilisés.

RIOT cible les plateformes disposant de configurations matérielles très
contraintes, c'est-à-dire des microcontrôleurs ayant quelques kilo-octets de
mémoire RAM et flash, et sur lesquels des systèmes tels que Linux ou Windows 
ne peuvent pas être utilisés. RIOT a été développé en suivant une architecture
dite "micro-noyau" ce qui lui permet d'avoir une empreinte mémoire et une
utilisation du processeur très faibles. RIOT s'appuie également sur un
ordonnanceur déterministe et "tickless", ce qui lui garantit d'une part une très
basse consommation énergétique et d'autres des capacités d'exécution de tâches
en temps réel.\\

Actuellement, RIOT supporte des microcontrôleurs 8, 16 et 32 bits, et des
architectures AVR (Atmel) et ARM (Atmel, STMicro, Texas Instruments).

\subsection{Contexte scientifique}

L'ADT se déroule au sein de l'équipe-projet INFINE (INFormatIon NEtworks) dont
la recherche s'articule autour de 3 axes principaux:

\begin{enumerate}
	\item Online Social Networks
	\begin{itemize}
		\item Recommandation et vie privée
	\end{itemize}
	\item Traffic Offloading
	\begin{itemize}
		\item Dans le coeur de l'Internet
		\item Aux limites de l'Internet
	\end{itemize}
	\item IoT and Spontaneous Wireless Networks
	\begin{itemize}
		\item Support pour les petits "objets" dans les protocoles IP
	\end{itemize}
\end{enumerate}

Cette ADT est donc concernée par ce troisème axe qui vise à proposer des
protocoles réseau et des architectures optimisés et adaptés aux nouveaux
concepts de communication (e.g publish/subscribe, etc). Dans le cadre de
l'internet des objets, le système d'exploitation RIOT est utilisé comme support
logiciel\cite{RIOT-infocom} pour la mise en oeuvre des avancées liées à cet axe de recherche.
En particulier, plusieurs thèmes sont explorés:
\begin{itemize}
    \item Le développement de protocoles IP tolérant les coupures réseau des LLN (Low
power and Lossy Networks),
    \item Les réseaux ICN (Information Centric Networking)\cite{ICN-RIOT} dont le but est de
traiter la localisation et la reproduction de contenu dans la pile réseau, sans
adressage IP,
    \item Le suivi des évolutions des protocoles de la pile réseau discutées à l'IETF
(Internet Engineering Task Force)
\end{itemize}

Cette ADT doit aussi servir à intégrer les résultats de mes précédents travaux
de recherches sur la mise jour de composants logiciels sur les dispositifs très
contraints\cite{models-at-runtime}, en utilisant les protocoles réseaux sans fil de l'IoT (Internet
Of Things), comme 802.15.4 ou LoRa (Long Range). On parle alors de mise-à-jour
"Over the Air" (OTA).



\subsection{Objectifs de l'ADT}

L'objectif principal de l'ADT est le développement et la maintenance du code de
ce système d'exploitation ainsi que l'animation de sa communauté de
développeurs.
Plus précisément, ces objectifs sont les suivants :
\begin{description}
    \item[Ajout de nouvelles plateformes :] RIOT supporte un grand nombre de
        plateformes
        matérielles\footnote{\url{https://github.com/RIOT-OS/RIOT/wiki/RIOT-Platforms}},
        cependant ce secteur est en perpétuelle évolution et de nouvelles plateformes
        sont disponibles régulièrement. Pour promouvoir l'utilisation et la 
        collaboration en la communauté de développeurs, les scientifiques, les 
        industriels et les hobbyistes/makers, il est nécessaire de maintenir un effort 
        pour suivre ces évolutions en ajoutant le support de ces nouvelles configurations.

    \item[Support des nouveaux protocoles réseau :] l'IETF discute régulièrement de
        l'amélioration ou de la définition de nouveaux protocoles réseaux, en
        particulier pour l'IoT. Un des enjeux principaux de cette ADT est donc
        d'implémenter ces nouveaux protocoles réseaux dans la pile protocolaire
        logicielle de RIOT et de les tester en conditions réelles, par exemple en
        utilisant les noeuds ouverts de la plateforme expérimentale IoT-LAB.

    \item[Mise à jour des composants logiciels de RIOT :] La mise à jour d'un système
        d'exploitation et de ses composants logiciels est essentielle pour garantir sa
        sécurité, corriger des bogues ou intégrer de nouvelles fonctionnalités. Dans le
        cas d'un système embarqué IoT fortement contraint, ces mises à jour représentent
        encore un défi technique très complexe: les réseaux sans-fil sont peu fiables et très bas débit,
        les équipements n'ont pas d'interface utilisateur accessible à distance et
        l'espace mémoire y est très limitée.

    \item[Revue de Code :] La revue de code consiste à vérifier que le code écrit par les
        contributeurs de la communauté fonctionne correctement, n'introduit pas de
        régression et qu'il respecte les conventions établies par les mainteneurs du
        projet. Les vérifications fonctionnelles s'effectuent soit à distance dans un
        environnement de test (appelé "testbed",  comme 
        IoT-LAB\footnote{\url{https://www.iot-lab.info}}) ou directement sur le matériel 
        pour lequel il a été développé.
\end{description}


%\begin{notes}
%  Contexte scientifique de l'équipe de recherche,
%  sujet des travaux, état initial \texttt{avant l'année en cours} et objectifs.
%  \\

%  \textit{Cette partie fait écho aux trois premières sections
%    de la demande d'ADT : \textbf{Introduction}, \textbf{Contexte : état des lieux et positionnement avant l'ADT} et \textbf{Objectifs de l'ADT}.}
%\end{notes}

%%==============================================================================
%%==============================================================================
\section{Travaux réalisés}


\begin{description}
    \item[Ajout de nouvelles plateformes :] Pendant ces 8 mois d'ADT, j'ai 
        ajouté dans RIOT le support d'une nouvelle plateforme ainsi qu'une 
        nouvelle interface radio utilisée par plusieurs cartes IoT. Par 
        ailleurs, j'ai proposé plusieurs corrections de bogues. Le détail de 
        ces développement est le suivant:
        \begin{itemize}
            \item Support de la plateforme Waspmote Pro de Libellium, basée 
                sur un microcontrôleur AVR 8 bit : j'ai écrit le code pour 
                supporter les périphériques de base (CPU, timers) et 
                implémenter les capacités de colmmunication (interfaces UART,
                SPI),

            \item Support de la puce radio Texas Instrument CC2420 qui équipe de 
                nombreuses plateformes: Zolertia Z1, TelosB, WSN430. Ce travail a permis de
                faire les expérimentations nécessaires pour l'écriture d'un article de 
                recherche comparant les performances de RIOT avec d'autres systèmes
                d'exploitation similaires (Contiki\footnote{\url{http://contiki-os.org}},
                OpenWSN, etc)
            \item J'ai également contribué et intégré plusieurs correction de bogues, en
                  particulier sur les plateformes AVR 8 bit. Ces contributions améliorent la
                  performance de RIOT sur ces architectures et ouvrent la voie pour
                  supporter de nouvelles plateformes comme par exemple la carte Arduino UNO.
        \end{itemize}

    \item[Revue de code :] Depuis le début de l'ADT, j'ai révisé et intégré dans la
        branche principale de RIOT le code de plus de 60 "pull-requests" (PR).
        Voici les plus importantes :
        \begin{itemize}
            \item PR\#4725\footnote{\url{https://github.com/RIOT-OS/RIOT/pull/4725}} 
                permettant de configurer un routeur de bordure ("Border
                Router") avec un seul lien série (initialement, il fallait utiliser 2
                liens série),
            \item PR\#5167\footnote{\url{https://github.com/RIOT-OS/RIOT/pull/5167}} 
                qui a étendu le support de RIOT sur les noeuds A8-M3 de la
                plateforme expérimentale IoT-LAB
            \item PR\#5291\footnote{\url{https://github.com/RIOT-OS/RIOT/pull/5291}} 
                ajoutant le support de la puce radio Texas Instrument CC2538
            \item Plusieurs PR améliorant le support des microcontrôleurs Atmel AVR
                ATMega, notamment utilisés sur les plateformes Arduino (Mega2560, Uno,
                DueMilanove)
            \item Plusieurs PR liées à l'ajout du support du protocole de sécurisation du
                transfert de données DTLS.
        \end{itemize}
    \item[Aspects communautaires:]
        Au bout de quelques mois après le début de l'ADT, ma participation, mes
        contributions et mon implication au projet RIOT étant très appréciées,
        j'ai intégré l'équipe des mainteneurs officiels. Je suis donc désormais
        autorisé à fusionner, après vérification, les contributions de la
        communauté des développeurs dans la branche principale et je participe aux
        décisions relatives à l'organisation du projet.
        Depuis le début de l'ADT, je participe également aux sprints mensuels
        appelés Hack'n'ACK et organisés par la communauté RIOT dans le but
        d'avancer rapidement, à plusieurs, sur des développement en cours et à
        accélérer l'intégration de PR en cours de vérification.\\

        De plus, j'ai aussi pris part très activement à l'organisation du RIOT
        Summit\footnote{\url{http://summit.riot-os.org}}, qui a réuni les 14 et 
        15 juillet à Berlin, plus de 100
        développeurs, chercheurs et industriels du domaine de l'IoT.
        J'ai contribué à l'écriture de l'article "The Future of IoT Software Must
        be Updated"\cite{IOTSU}, illustrant les principaux défis de la mise à 
        jour logicielle,
        et présenté lors d'un "workshop" organisé par l'IETF à Dublin (je croyais
        que c'était à Berlin, quelle date?).\\

        Enfin, j'ai été responsable d'organiser la sortie en juillet de la version
        2016.07 de RIOT. Ce rôle consistait à:
        \begin{itemize}
            \item trier les PRs prêtes ou quasiment prêtes pour faire partie de cette
                version,
            \item faire des vérification de PRs
            \item accélérer l'intégration des nouveaux bogues qui pouvaient apparaître.
        \end{itemize}
        Voici quelques données concernant cette version:
        \begin{itemize}
            \item 198 PR et contenant 325 "commits" ont été intégrées. 65 tickets
                ("issues") ont ainsi été résolus,
            \item 46 personnes ont contribué du code en 112 jours
            \item 632 fichiers sources ont été modifiés ou ajoutés, représentant 19863
                lignes insérées et 3682 lignes supprimées.
        \end{itemize}
\end{description}

%\begin{notes}
%  Travaux réalisés \texttt{pendant l'année en cours},
%  relativement aux jalons et tâches prévus dans la planification initiale.
%  \\

%  \textit{Cette partie fait écho à la cinquième section de la demande d'ADT :
%    \textbf{Mise en oeuvre prévisionnelle de l'ADT.}}
%\end{notes}

%%==============================================================================
%%==============================================================================
\section{Travaux restants / perspectives}
Les perspectives pour la fin de cette année et la deuxième année de travail sont les suivantes:
\begin{enumerate}
	\item Finalisation de la vérification du code pour la PR de TinyDTLS.
	\item Rédaction d'un plan de travail pour le support du protocole MQTT.
	\item Mise en oeuvre d'une plateforme de mise à jour des composants logiciels de RIOT dans le testbed IoT-Lab.
	\item Continuer les tâches de révision de code, organisation et tous les activités au sein de la communauté RIOT.
\end{enumerate}

%\begin{notes}
%  Si l'année en cours est la première année de contrat :
%  plan de travail \texttt{pour la deuxième année},
%  relativement aux jalons et tâches prévus dans la planification initiale.
%  \\

%  \textit{Cette partie fait écho à la quatrième section de la demande d'ADT :
%    \textbf{Mise en oeuvre prévisionnelle de l'ADT.}}
%  \\

%  Si l'année en cours est la deuxième année de contrat :
%  adéquation des travaux réalisés \texttt{sur les deux années},
%  relativement à la sortie prévue initialement et perspectives générales.
%  \\

%  \textit{Cette partie fait écho à la quatrième section de la demande d'ADT :
%    \textbf{Sortie de l'ADT : positionnement après l'ADT.}}

%\end{notes}

%%==============================================================================
%%==============================================================================
\section{Bilan}
Jusqu'à ce jour, mes contributions dans l'ADT sont les suivantes:

\begin{itemize}
	\item 19 commits / 3,536 ++ / 2,184 --
	\item Participation dans l'organisation du RIOT Summit 2016.
	\item 1 article scientifique.
	\item Revue très active du code avec plus 60 PRs migrés.
\end{itemize}

\subsection*{Bilan personnel}
D'un point de vue personnel, je suis très satisfait de mon travail au sein de
l'équipe INFINE où j'ai été très bien intégré. Les échanges virtuels, à travers
la plateforme de développement GitHub, et personnels avec la communauté RIOT à
Berlin ont été très enrichissants, tant au niveau professionnel, académique que
personnel.

La participation à l'IETF 96 et l'organisation du RIOT Summit 2016 m'ont aussi permis d'échanger plusieurs points de vue vis-à-vis du domaine de recherche de l'Internet des Objets, axe que je voudrais approfondir dans mes futures activités dans cette ADT.\\

Les lieux de travail et l'attention du personnel du CRI Saclay, en particulier
le SED, m'ont aussi permis d'exploiter au maximum mes compétences et d'en
développer de nouvelles en très peu de temps.


%\begin{notes}
%  Mesures de l'avancement de l'ADT,
%  relativement à l'évaluation initialement proposée.
%  \\

%  \textit{Cette partie fait écho à la septième section de la demande d'ADT :
%    \textbf{Suivi et évaluation}}.
%  \\

%  Discussion sur le déroulement de l'ADT :
%  expression des besoins, changements survenus, retours de tests.
%  \\

%  Bilan personnel :
%  apport de l'expérience à l'INRIA, compétences acquises ou développées.

%\end{notes}

%%==============================================================================
%%==============================================================================
\section{Annexes}

%===============================================================================
\subsection{Prise en main du logiciel}
Liste de langages de développement utilisés :

\begin{itemize}
	\item C, C++
	\item Python
	\item Make\\
\end{itemize}

Liste de systèmes d'exploitation :

\begin{itemize}
	\item Mac OSX El Capitan
	\item Linux
	\item Windows
	\item RIOT
	\item Contiki\\
\end{itemize}

Compilateurs :
\begin{itemize}
	\item GCC and GCC for ARM, AVR and msp430
	\item LLVM (from Mac toolchains)\\
\end{itemize}

La description de l'architecture logiciel ainsi que les détails sur les dépendances se trouvent sur \url{https://github.com/RIOT-OS/RIOT} et \url{https://github.com/RIOT-OS/RIOT/wiki}

%\begin{notes}

%  \begin{itemize}

%  \item Liste des langages de développement utilisés,
%    systèmes d'exploitation cibles, compilateurs, livrables produits

%  \item Pointeur vers
%    une description de l'architecture logicielle (différents composants),
%    description des dépendances (licences, versions etc.)

%  \item Localisation des sources, pointeur vers un guide de compilation,
%    pointeur vers un guide d'exécution des tests,
%    pointeur vers un guide d'extraction de la documentation de code,
%    localisation du gestionnaire de suivi (bogues, fonctionnalités etc.)

%  \end{itemize}

%\end{notes}

%===============================================================================
\section{Documentation}
Toute la documentation sur RIOT se trouve ici : \\

\url{http://riot-os.org/api/}

%\begin{notes}
%  Pointeurs vers

%  \begin{itemize}

%  \item Le manuel de référence

%  \item Le tutoriel

%  \item Le manuel utilisateur

%  \item Le manuel de maintenance pour le développeur

%  \end{itemize}

%\end{notes}

\begin{thebibliography}{1}
%	\bibitem{RFC7228} C. Bormann, M. Ersue, A Keranen. "Terminology for constrained-node networks." Internet Engineering Task Force (IETF), RFC 7228, May 2014.
%	\bibitem{IoT-OS-Survey} O. Hahm, E. Baccelli, H. Petersen, N. Tsiftes, ''Operating Systems for Low-End Devices in the Internet of Things : a Survey,'' in IEEE Internet of Things Journal, Dec. 2015.
%	\bibitem{Schneier} B. Schneier. "The Internet of things is wildly insecure -- and often unpatchable." Schneier on Security \url{https://www.schneier.com/essays/archives/2014/01/the_internet_of_thin.html} January 2014.
%	\bibitem{darjeeling}Brouwers, N., Langendoen, K., \& Corke, P. "Darjeeling, a feature-rich VM for the resource poor." In Proceedings of ACM SenSys. Nov. 2009.
%	\bibitem{WIRED} B. Barrett, "Wink?s Outage Shows Us How Frustrating Smart Homes Could Be." WIRED, April 2015. \url{http://www.wired.com/2015/04/smart-home-headaches/}
%	\bibitem{RIOT-ercim} E. Baccelli, O. Hahm, H. Petersen, and K. Schleiser. ''RIOT and the Evolution of IoT Operating Systems and Applications.'' ERCIM News no. 101, April 2015.
%	\bibitem{lejos} Solorzano, J. "leJOS: Java based OS for Lego RCX." Online at: \url{http://lejos.sourceforge.net}
	\bibitem{RIOT-infocom} E. Baccelli, O. Hahm, M. W{\"a}hlisch, M. G{\"u}nes, T. Schmidt, ''RIOT OS: Towards an OS for the Internet of Things,'' in Proceedings of IEEE INFOCOM, April 2013.
	\bibitem{IOTSU} F. Acosta, E. Baccelli, T. Eichinger, K. Schleiser, "The Future of IoT Software Must be Updated", In IoT Software Update Workshop 2016, June 2016
%	\bibitem{reprog-runtime}Oliver, R., Wilde, A., \& Zaluska, E. (2014). "Reprogramming embedded systems at run-time." In Proceedings of IEEE ICST, Sept. 2014
%	\bibitem{SDZ} H. Tanriverdi, "Als eine Gl\"uhbirne das Smart Home lahmlegte." S\"uddeutsche Zeitung, March 2015. 
%	\url{http://www.sueddeutsche.de/digital/dos-attacke-als-eine-gluehbirne-das-smart-home-lahmlegte-1.2380844}
%	\bibitem{TrendMicro} Trend Micro Report, "Researchers Discover a Not-So-Smart Flaw In Smart Toy Bear", February 2016 \url{http://www.trendmicro.com/vinfo/us/security/news/internet-of-things/researchers-discover-flaw-in-smart-toy-bear}
%	\bibitem{Nest-bricking} A. Gilbert. "The time that Tony Fadell sold me a container of hummus". \url{https://arlogilbert.com/the-time-that-tony-fadell-sold-me-a-container-of-hummus-cb0941c762c1}
	\bibitem{models-at-runtime}FJ. Acosta Padilla, F. Weis, J. Bourcier. "Towards a model@ runtime middleware for cyber physical systems." Proceedings of ACM / USENIX DeDiSys, 2014.
%	\bibitem{ThingSquare} Why You Want Firmware Updates, ThingSquare Blog, \url{http://www.thingsquare.com/blog/articles/firmware-updates/}
%	\bibitem{deluge}Hui, Jonathan W., and David Culler. "The dynamic behavior of a data dissemination protocol for network programming at scale." Proceedings of the 2nd international conference on Embedded networked sensor systems. ACM, 2004.
%	\bibitem{figaro}Mottola, Luca, Gian Pietro Picco, and Adil Amjad Sheikh. "FiGaRo: fine-grained software reconfiguration for wireless sensor networks." Wireless Sensor Networks. Springer Berlin Heidelberg, 2008. 286-304.
	\bibitem{ICN-RIOT} "Information Centric Networking in the IoT: Experiments with NDN in the Wild" 
	E. Baccelli, C. Mehlis, O. Hahm, T. C. Schmidt, and M. Wählisch ACM ICN, 2014.
%	\bibitem{nxp-sec-ic} NXP, "A710x family: Secure authentication microcontroller", May 2016 \url{http://www.nxp.com/products/identification-and-security/authentication/secure-authentication-microcontroller:A710X_FAMILY}
%	\bibitem{inf-sec-ic} Infineon Technologies AG, "OPTIGA TRUST P SLJ 52ACA", May 2016 \url{http://www.infineon.com/cms/en/product/productType.html?productType=5546d4624f205c9a014f6eec8c007b9a}
\end{thebibliography} 

\end{document}
