\documentclass[a4paper, twoside]{article}

\usepackage[T1]{fontenc}
\usepackage{fullpage}
\usepackage[french]{babel}
\usepackage[latin1]{inputenc}
\usepackage{color}
\usepackage{hyperref}

\newenvironment{notes}[0]{\color{blue}}{\color{black}}

\title{\vskip 5cm {\huge
    Rapport d'activit\'e
      %~\footnote{}
      %/ Rapport final
      %~\footnote{Si l'ann\'ee en cours est la deuxi\`eme ann\'ee de contrat}
      \\
      D\'eveloppement open-source et maintenance d'un OS embarqu\'e pour l'Internet des objets
  }
}
\author{\large \sl \bf
  Francisco Javier Acosta Padilla \\
  Ing\'enieur sp\'ecialiste
}
\date{
  Du 1 janvier 2016 au 26 septembre 2016
}

\begin{document}
\maketitle

\newpage

%%==============================================================================
%%==============================================================================
\section{Contexte}


\subsection{Introduction}


Cette ADT a \'et\'e lanc\'ee dans le cadre du projet de d\'eveloppement d'un syst\`eme
d'exploitation pour l'internet des objets, RIOT.\\

RIOT a commenc� en 2013, fond\'e par INRIA, Freie Universit\"at Berlin et Hochschule f\"ur Angewandte Wissenschaften Hamburg, et compte \`a ce jour plus de 120 contributeurs de code: des d\'eveloppeurs venant du monde entier.

Le code source est disponible sur
Github\footnote{\url{https://github.com/RIOT-OS/RIOT/}}, sous licence LGLPv2,
qui permet \`a� toute personne son utilisation, modification et distribution.

Ce code est essentiellement \'ecrit en C et utilise l'outil
Make pour la compilation. D'autres langages comme Python ou C++ sont \'egalement
utilis\'es.

RIOT cible les plateformes disposant de configurations mat\'erielles tr\`es
contraintes, c'est-\`a-dire des microcontr\^oleurs ayant quelques kilo-octets de
m\'emoire RAM et flash, et sur lesquels des syst\`emes tels que Linux ou Windows 
ne peuvent pas \^etre utilis\'es. RIOT a \'et\'e d\'evelopp\'e en suivant une architecture
dite "micro-noyau" ce qui lui permet d'avoir une empreinte m\'emoire et une
utilisation du processeur tr\`es faibles. RIOT s'appuie \'egalement sur un
ordonnanceur d\'eterministe et "tickless", ce qui lui garantit d'une part une tr\`es
basse consommation \'energ\'etique et d'autres des capacit\'es d'ex\'ecution de t\^aches
en temps r\'eel.\\

Actuellement, RIOT supporte des microcontr\^oleurs 8, 16 et 32 bits, et des
architectures AVR (Atmel) et ARM (Atmel, STMicro, Texas Instruments).

\subsection{Contexte scientifique}

L'ADT se d\'eroule au sein de l'\'equipe-projet INFINE (INFormatIon NEtworks) dont
la recherche s'articule autour de 3 axes principaux:

\begin{enumerate}
	\item Online Social Networks
	\begin{itemize}
		\item Recommandation et vie priv\'ee
	\end{itemize}
	\item Traffic Offloading
	\begin{itemize}
		\item Dans le coeur de l'Internet
		\item Aux limites de l'Internet
	\end{itemize}
	\item IoT and Spontaneous Wireless Networks
	\begin{itemize}
		\item Support pour les petits "objets" dans les protocoles IP
	\end{itemize}
\end{enumerate}

Cette ADT est donc concern\'ee par ce trois\`eme axe qui vise \`a� proposer des
protocoles r\'eseau et des architectures optimis\'es et adapt\'es aux nouveaux
concepts de communication (e.g publish/subscribe, etc). Dans le cadre de
l'internet des objets, le syst\`eme d'exploitation RIOT est utilis\'e comme support
logiciel\cite{RIOT-infocom} pour la mise en oeuvre des avanc\'ees li\'ees \`a� cet axe de recherche.
En particulier, plusieurs th\`emes sont explor\'es:
\begin{itemize}
    \item Le d\'eveloppement de protocoles IP tol\'erant les coupures r\'eseau des LLN (Low
power and Lossy Networks),
    \item Les r\'eseaux ICN (Information Centric Networking)\cite{ICN-RIOT} dont le but est de
traiter la localisation et la reproduction de contenu dans la pile r\'eseau, sans
adressage IP,
    \item Le suivi des \'evolutions des protocoles de la pile r\'eseau discut\'ees \`a� l'IETF
(Internet Engineering Task Force)
\end{itemize}

Cette ADT doit aussi servir \`a�int\'egrer les r\'esultats de mes pr\'ec\'edents travaux
de recherches sur la mise jour de composants logiciels sur les dispositifs tr\`es
contraints\cite{models-at-runtime}, en utilisant les protocoles r\'eseaux sans fil de l'IoT (Internet
Of Things), comme 802.15.4 ou LoRa (Long Range). On parle alors de mise-\`a-jour
"Over the Air" (OTA).



\subsection{Objectifs de l'ADT}

L'objectif principal de l'ADT est le d\'eveloppement et la maintenance du code de
ce syst\`eme d'exploitation ainsi que l'animation de sa communaut\'e de
d\'eveloppeurs.
Plus pr\'ecis\'ement, ces objectifs sont les suivants :
\begin{description}
    \item[Ajout de nouvelles plateformes :] RIOT supporte un grand nombre de
        plateformes
        mat\'erielles\footnote{\url{https://github.com/RIOT-OS/RIOT/wiki/RIOT-Platforms}},
        cependant ce secteur est en perp\'etuelle \'evolution et de nouvelles plateformes
        sont disponibles r\'eguli\`erement. Pour promouvoir l'utilisation et la 
        collaboration en la communaut\'e de d\'eveloppeurs, les scientifiques, les 
        industriels et les hobbyistes/makers, il est n\'ecessaire de maintenir un effort 
        pour suivre ces \'evolutions en ajoutant le support de ces nouvelles configurations.

    \item[Support des nouveaux protocoles r\'eseau :] l'IETF discute r\'eguli\`erement de
        l'am\'elioration ou de la d\'efinition de nouveaux protocoles r\'eseaux, en
        particulier pour l'IoT. Un des enjeux principaux de cette ADT est donc
        d'impl\'ementer ces nouveaux protocoles r\'eseaux dans la pile protocolaire
        logicielle de RIOT et de les tester en conditions r\'eelles, par exemple en
        utilisant les noeuds ouverts de la plateforme exp\'erimentale IoT-LAB.

    \item[Mise \`a� jour des composants logiciels de RIOT :] La mise \`a� jour d'un syst\`eme
        d'exploitation et de ses composants logiciels est essentielle pour garantir sa
        s\'ecurit\'e, corriger des bogues ou int\'egrer de nouvelles fonctionnalit\'es. Dans le
        cas d'un syst\`eme embarqu\'e IoT fortement contraint, ces mises \`a� jour repr\'esentent
        encore un d\'efi technique tr\`es complexe: les r\'eseaux sans-fil sont peu fiables et tr\`es bas d\'ebit,
        les \'equipements n'ont pas d'interface utilisateur accessible \`a distance et
        l'espace m\'emoire y est tr\`es limit\'ee.

    \item[Revue de Code :] La revue de code consiste \`a� v\'erifier que le code \'ecrit par les
        contributeurs de la communaut\'e fonctionne correctement, n'introduit pas de
        r\'egression et qu'il respecte les conventions \'etablies par les mainteneurs du
        projet. Les v\'erifications fonctionnelles s'effectuent soit \`a� distance dans un
        environnement de test (appel\'e "testbed",  comme 
        IoT-LAB\footnote{\url{https://www.iot-lab.info}}) ou directement sur le mat\'eriel 
        pour lequel il a \'et\'e d\'evelopp\'e.
\end{description}


%\begin{notes}
%  Contexte scientifique de l'\'equipe de recherche,
%  sujet des travaux, \'etat initial \texttt{avant l'ann\'ee en cours} et objectifs.
%  \\

%  \textit{Cette partie fait \'echo aux trois premi\`eres sections
%    de la demande d'ADT : \textbf{Introduction}, \textbf{Contexte : \'etat des lieux et positionnement avant l'ADT} et \textbf{Objectifs de l'ADT}.}
%\end{notes}

%%==============================================================================
%%==============================================================================
\section{Travaux r\'ealis\'es}


\begin{description}
    \item[Ajout de nouvelles plateformes :] Pendant ces 8 mois d'ADT, j'ai 
        ajout\'e dans RIOT le support d'une nouvelle plateforme ainsi qu'une 
        nouvelle interface radio utilis\'ee par plusieurs cartes IoT. Par 
        ailleurs, j'ai propos\'e plusieurs corrections de bogues. Le d\'etail de 
        ces d\'eveloppement est le suivant:
        \begin{itemize}
            \item Support de la plateforme Waspmote Pro de Libellium, bas\'ee 
                sur un microcontr\^oleur AVR 8 bit : j'ai \'ecrit le code pour 
                supporter les p\'eriph\'eriques de base (CPU, timers) et 
                impl\'ementer les capacit\'es de colmmunication (interfaces UART,
                SPI),

            \item Support de la puce radio Texas Instrument CC2420 qui \'equipe de 
                nombreuses plateformes: Zolertia Z1, TelosB, WSN430. Ce travail a permis de
                faire les exp\'erimentations n\'ecessaires pour l'\'ecriture d'un article de 
                recherche comparant les performances de RIOT avec d'autres syst\`emes
                d'exploitation similaires (Contiki\footnote{\url{http://contiki-os.org}},
                OpenWSN, etc)
            \item J'ai \'egalement contribu\'e et int\'egr\'e plusieurs correction de bogues, en
                  particulier sur les plateformes AVR 8 bit. Ces contributions am\'eliorent la
                  performance de RIOT sur ces architectures et ouvrent la voie pour
                  supporter de nouvelles plateformes comme par exemple la carte Arduino UNO.
        \end{itemize}

    \item[Revue de code :] Depuis le d\'ebut de l'ADT, j'ai r\'evis\'e et int\'egr\'e dans la
        branche principale de RIOT le code de plus de 60 "pull-requests" (PR).
        Voici les plus importantes :
        \begin{itemize}
            \item PR\#4725\footnote{\url{https://github.com/RIOT-OS/RIOT/pull/4725}} 
                permettant de configurer un routeur de bordure ("Border
                Router") avec un seul lien s\'erie (initialement, il fallait utiliser 2
                liens s\'erie),
            \item PR\#5167\footnote{\url{https://github.com/RIOT-OS/RIOT/pull/5167}} 
                qui a \'etendu le support de RIOT sur les noeuds A8-M3 de la
                plateforme exp\'erimentale IoT-LAB
            \item PR\#5291\footnote{\url{https://github.com/RIOT-OS/RIOT/pull/5291}} 
                ajoutant le support de la puce radio Texas Instrument CC2538
            \item Plusieurs PR am\'eliorant le support des microcontr\^oleurs Atmel AVR
                ATMega, notamment utilis\'es sur les plateformes Arduino (Mega2560, Uno,
                DueMilanove)
            \item Plusieurs PR li\'ees \`a l'ajout du support du protocole de s\'ecurisation du
                transfert de donn\'ees DTLS.
        \end{itemize}
    \item[Aspects communautaires:]
        Au bout de quelques mois apr\`es le d\'ebut de l'ADT, ma participation, mes
        contributions et mon implication au projet RIOT \'etant tr\`es appr\'eci\'ees,
        j'ai int\'egr\'e l'\'equipe des mainteneurs officiels. Je suis donc d\'esormais
        autoris\'e \`a� fusionner, apr\`es v\'erification, les contributions de la
        communaut\'e des d\'eveloppeurs dans la branche principale et je participe aux
        d\'ecisions relatives \`a� l'organisation du projet.
        Depuis le d\'ebut de l'ADT, je participe \'egalement aux sprints mensuels
        appel\'es Hack'n'ACK et organis\'es par la communaut\'e RIOT dans le but
        d'avancer rapidement, \`a plusieurs, sur des d\'eveloppements en cours et \`a
        acc\'el\'erer l'int\'egration de PR en cours de v\'erification.\\

        De plus, j'ai aussi pris part tr\`es activement \`a� l'organisation du RIOT
        Summit\footnote{\url{http://summit.riot-os.org}}, qui a r\'euni les 14 et 
        15 juillet \`a� Berlin, plus de 100
        d\'eveloppeurs, chercheurs et industriels du domaine de l'IoT.
        J'ai contribu\'e \`a� l'\'ecriture de l'article "The Future of IoT Software Must
        be Updated"\cite{IOTSU}, illustrant les principaux d\'efis de la mise \`a� 
        jour logicielle,
        et pr\'esent\'e lors d'un "workshop" organis\'e par l'IETF \`a Dublin le 13 et 14 juin\\

        Enfin, j'ai \'et\'e responsable d'organiser la sortie en juillet de la version
        2016.07 de RIOT. Ce r\^ole consistait \`a�:
        \begin{itemize}
            \item trier les PRs pr\^etes ou quasiment pr\^etes pour faire partie de cette version,
            \item faire des v\'erification de PRs
            \item acc\'el\'erer l'int\'egration des nouveaux bogues qui pouvaient appara\^itre.
        \end{itemize}
        Voici quelques donn\'ees concernant cette version:
        \begin{itemize}
            \item 198 PR et contenant 325 "commits" ont \'et\'e int\'egr\'ees. 65 tickets
                ("issues") ont ainsi \'et\'e r\'esolus,
            \item 46 personnes ont contribu\'e du code en 112 jours
            \item 632 fichiers sources ont \'et\'e modifi\'es ou ajout\'es, repr\'esentant 19863
                lignes ins\'er\'ees et 3682 lignes supprim\'ees.
        \end{itemize}
\end{description}

%\begin{notes}
%  Travaux r\'ealis\'es \texttt{pendant l'ann\'ee en cours},
%  relativement aux jalons et t\`a�ches pr\'evus dans la planification initiale.
%  \\

%  \textit{Cette partie fait \'echo \`a� la cinqui\`eme section de la demande d'ADT :
%    \textbf{Mise en oeuvre pr\'evisionnelle de l'ADT.}}
%\end{notes}

%%==============================================================================
%%==============================================================================
\section{Travaux restants / perspectives}
Les perspectives pour la fin de cette ann\'ee et la deuxi\`eme ann\'ee de travail sont les suivantes:
\begin{enumerate}
	\item Finalisation de la v\'erification du code pour la PR de TinyDTLS.
	\item R\'edaction d'un plan de travail pour le support du protocole MQTT.
	\item Mise en oeuvre d'une plateforme de mise \`a� jour des composants logiciels de RIOT dans le testbed IoT-Lab.
	\item Continuer les t\^aches de r\'evision de code, organisation et tous les activit\'es au sein de la communaut\'e RIOT.
\end{enumerate}

%\begin{notes}
%  Si l'ann\'ee en cours est la premi\`ere ann\'ee de contrat :
%  plan de travail \texttt{pour la deuxi\`eme ann\'ee},
%  relativement aux jalons et t\`a�ches pr\'evus dans la planification initiale.
%  \\

%  \textit{Cette partie fait \'echo \`a� la quatri\`eme section de la demande d'ADT :
%    \textbf{Mise en oeuvre pr\'evisionnelle de l'ADT.}}
%  \\

%  Si l'ann\'ee en cours est la deuxi\`eme ann\'ee de contrat :
%  ad\'equation des travaux r\'ealis\'es \texttt{sur les deux ann\'ees},
%  relativement \`a� la sortie pr\'evue initialement et perspectives g\'en\'erales.
%  \\

%  \textit{Cette partie fait \'echo \`a� la quatri\`eme section de la demande d'ADT :
%    \textbf{Sortie de l'ADT : positionnement apr\`es l'ADT.}}

%\end{notes}

%%==============================================================================
%%==============================================================================
\section{Bilan}
Jusqu'\`a� ce jour, mes contributions dans l'ADT sont les suivantes:

\begin{itemize}
	\item 19 commits / 3,536 ++ / 2,184 --
	\item Participation dans l'organisation du RIOT Summit 2016.
	\item 1 article scientifique.
	\item Revue tr\`es active du code avec plus de 60 PRs migr\'es.
\end{itemize}

\subsection*{Bilan personnel}
D'un point de vue personnel, je suis tr\`es satisfait de mon travail au sein de
l'\'equipe INFINE o\`u j'ai \'et\'e tr\`es bien int\'egr\'e. Les \'echanges virtuels, \`a� travers
la plateforme de d\'eveloppement GitHub, et personnels avec la communaut\'e RIOT \`a�
Berlin ont \'et\'e tr\`es enrichissants, tant au niveau professionnel, acad\'emique que
personnel.

La participation \`a�l'IETF 96 et l'organisation du RIOT Summit 2016 m'ont aussi permis d'\'echanger plusieurs points de vue vis-\`a-vis du domaine de recherche de l'Internet des Objets, axe que je voudrais approfondir dans mes futures activit\'es dans cette ADT.\\

Les lieux de travail et l'attention du personnel du CRI Saclay, en particulier
le SED, m'ont aussi permis d'exploiter au maximum mes comp\'etences et d'en
d\'evelopper de nouvelles en tr\`es peu de temps.


%\begin{notes}
%  Mesures de l'avancement de l'ADT,
%  relativement \`a� l'\'evaluation initialement propos\'ee.
%  \\

%  \textit{Cette partie fait \'echo \`a� la septi\`eme section de la demande d'ADT :
%    \textbf{Suivi et \'evaluation}}.
%  \\

%  Discussion sur le d\'eroulement de l'ADT :
%  expression des besoins, changements survenus, retours de tests.
%  \\

%  Bilan personnel :
%  apport de l'exp\'erience \`a� l'INRIA, comp\'etences acquises ou d\'evelopp\'ees.

%\end{notes}

%%==============================================================================
%%==============================================================================
\section{Annexes}

%===============================================================================
\subsection{Prise en main du logiciel}
Liste de langages de d\'eveloppement utilis\'es :

\begin{itemize}
	\item C, C++
	\item Python
	\item Make\\
\end{itemize}

Liste de syst\`emes d'exploitation :

\begin{itemize}
	\item Mac OSX El Capitan
	\item Linux
	\item Windows
	\item RIOT
	\item Contiki\\
\end{itemize}

Compilateurs :
\begin{itemize}
	\item GCC and GCC for ARM, AVR and msp430
	\item LLVM (from Mac toolchains)\\
\end{itemize}

La description de l'architecture logiciel ainsi que les d\'etails sur les d\'ependances se trouvent sur \url{https://github.com/RIOT-OS/RIOT} et \url{https://github.com/RIOT-OS/RIOT/wiki}

%\begin{notes}

%  \begin{itemize}

%  \item Liste des langages de d\'eveloppement utilis\'es,
%    syst\`emes d'exploitation cibles, compilateurs, livrables produits

%  \item Pointeur vers
%    une description de l'architecture logicielle (diff\'erents composants),
%    description des d\'ependances (licences, versions etc.)

%  \item Localisation des sources, pointeur vers un guide de compilation,
%    pointeur vers un guide d'ex\'ecution des tests,
%    pointeur vers un guide d'extraction de la documentation de code,
%    localisation du gestionnaire de suivi (bogues, fonctionnalit\'es etc.)

%  \end{itemize}

%\end{notes}

%===============================================================================
\section{Documentation}
Toute la documentation sur RIOT se trouve ici : \\

\url{http://riot-os.org/api/}

%\begin{notes}
%  Pointeurs vers

%  \begin{itemize}

%  \item Le manuel de r\'ef\'erence

%  \item Le tutoriel

%  \item Le manuel utilisateur

%  \item Le manuel de maintenance pour le d\'eveloppeur

%  \end{itemize}

%\end{notes}

\begin{thebibliography}{1}
%	\bibitem{RFC7228} C. Bormann, M. Ersue, A Keranen. "Terminology for constrained-node networks." Internet Engineering Task Force (IETF), RFC 7228, May 2014.
%	\bibitem{IoT-OS-Survey} O. Hahm, E. Baccelli, H. Petersen, N. Tsiftes, ''Operating Systems for Low-End Devices in the Internet of Things : a Survey,'' in IEEE Internet of Things Journal, Dec. 2015.
%	\bibitem{Schneier} B. Schneier. "The Internet of things is wildly insecure -- and often unpatchable." Schneier on Security \url{https://www.schneier.com/essays/archives/2014/01/the_internet_of_thin.html} January 2014.
%	\bibitem{darjeeling}Brouwers, N., Langendoen, K., \& Corke, P. "Darjeeling, a feature-rich VM for the resource poor." In Proceedings of ACM SenSys. Nov. 2009.
%	\bibitem{WIRED} B. Barrett, "Wink?s Outage Shows Us How Frustrating Smart Homes Could Be." WIRED, April 2015. \url{http://www.wired.com/2015/04/smart-home-headaches/}
%	\bibitem{RIOT-ercim} E. Baccelli, O. Hahm, H. Petersen, and K. Schleiser. ''RIOT and the Evolution of IoT Operating Systems and Applications.'' ERCIM News no. 101, April 2015.
%	\bibitem{lejos} Solorzano, J. "leJOS: Java based OS for Lego RCX." Online at: \url{http://lejos.sourceforge.net}
	\bibitem{RIOT-infocom} E. Baccelli, O. Hahm, M. W{\"a}hlisch, M. G{\"u}nes, T. Schmidt, ''RIOT OS: Towards an OS for the Internet of Things,'' in Proceedings of IEEE INFOCOM, April 2013.
	\bibitem{IOTSU} F. Acosta, E. Baccelli, T. Eichinger, K. Schleiser, "The Future of IoT Software Must be Updated", In IoT Software Update Workshop 2016, June 2016
%	\bibitem{reprog-runtime}Oliver, R., Wilde, A., \& Zaluska, E. (2014). "Reprogramming embedded systems at run-time." In Proceedings of IEEE ICST, Sept. 2014
%	\bibitem{SDZ} H. Tanriverdi, "Als eine Gl\"uhbirne das Smart Home lahmlegte." S\"uddeutsche Zeitung, March 2015. 
%	\url{http://www.sueddeutsche.de/digital/dos-attacke-als-eine-gluehbirne-das-smart-home-lahmlegte-1.2380844}
%	\bibitem{TrendMicro} Trend Micro Report, "Researchers Discover a Not-So-Smart Flaw In Smart Toy Bear", February 2016 \url{http://www.trendmicro.com/vinfo/us/security/news/internet-of-things/researchers-discover-flaw-in-smart-toy-bear}
%	\bibitem{Nest-bricking} A. Gilbert. "The time that Tony Fadell sold me a container of hummus". \url{https://arlogilbert.com/the-time-that-tony-fadell-sold-me-a-container-of-hummus-cb0941c762c1}
	\bibitem{models-at-runtime}FJ. Acosta Padilla, F. Weis, J. Bourcier. "Towards a model@ runtime middleware for cyber physical systems." Proceedings of ACM / USENIX DeDiSys, 2014.
%	\bibitem{ThingSquare} Why You Want Firmware Updates, ThingSquare Blog, \url{http://www.thingsquare.com/blog/articles/firmware-updates/}
%	\bibitem{deluge}Hui, Jonathan W., and David Culler. "The dynamic behavior of a data dissemination protocol for network programming at scale." Proceedings of the 2nd international conference on Embedded networked sensor systems. ACM, 2004.
%	\bibitem{figaro}Mottola, Luca, Gian Pietro Picco, and Adil Amjad Sheikh. "FiGaRo: fine-grained software reconfiguration for wireless sensor networks." Wireless Sensor Networks. Springer Berlin Heidelberg, 2008. 286-304.
	\bibitem{ICN-RIOT} "Information Centric Networking in the IoT: Experiments with NDN in the Wild" 
	E. Baccelli, C. Mehlis, O. Hahm, T. C. Schmidt, and M. W\"ahlisch ACM ICN, 2014.
%	\bibitem{nxp-sec-ic} NXP, "A710x family: Secure authentication microcontroller", May 2016 \url{http://www.nxp.com/products/identification-and-security/authentication/secure-authentication-microcontroller:A710X_FAMILY}
%	\bibitem{inf-sec-ic} Infineon Technologies AG, "OPTIGA TRUST P SLJ 52ACA", May 2016 \url{http://www.infineon.com/cms/en/product/productType.html?productType=5546d4624f205c9a014f6eec8c007b9a}
\end{thebibliography} 

\end{document}
